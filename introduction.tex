\section{An Introduction to Mobile Edge Computing}  \label{intro}


Toward 5G networks and the fourth industries, Telco companies and service providers are facing with tremedous oppotunities and challenges. One of them is to support innovative mobile applications, which require new generations of smart devices (e.g., AR/VR and smartphone) that are quickly released. Such a phenomenon results in an explosion of data. The fact is that 63\% of the world population currently acquires a mobile subscription, meanwhile, it was just 20\% in the previous decade. In addition, new use cases such as Internet of Things (IoT) and Machine-Type-Communications (MTC) introduce a huge number of connections, which shows promising business opportunities for mobile operators.

Considering the potential benefits from processing the huge amount of data, there is a need to design a cloud architecture. However, to ease the service management, the current cloud design follows a conventional centralized way that introduces major limiations, which could hardly apply for the emerging use cases. One of the main weakness is that the centralized cloud architecture introduce significant execution delay.

a cloud architecture is required... however, there are still some problems with the current...


Why MEC? 
  - explosion of data, require new technique
  - Issues in data center -- low latency, traffic congestion,...

Mobile Edge Computing (MEC) was introducued to fill these gaps.

What MEC? draw a figure


\begin{figure}[H]
  \begin{center}
   \includegraphics[width=13cm]{./figures/mec-arch.pdf}
   \caption{An example of MEC architecture}
   \label{fig:mec-arch}
   \end{center}
\end{figure}
